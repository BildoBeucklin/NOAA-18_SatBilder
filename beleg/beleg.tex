\documentclass[12pt,a4paper]{scrartcl}
\usepackage[T1]{fontenc}
\usepackage[latin1]{inputenc}
\usepackage{graphicx}
\usepackage{hyperref}
\usepackage{fancyref}
\usepackage[ngerman]{babel}
\usepackage{times}
\usepackage{listings}





\title{Empfangen von Satellitenbildern durch RTL SDR} 
\author{\\\\\\\\\\\\\\\\\\\\\\\\\\\\\\\\\
  Moritz Lechner\\
  Konstantin Ro\ss mann\\
  Leon Sobotta\\\\
  Seminar Advanced Computer Systems\\
  Computer Engineering
}


\date{\today}


\begin{document}


\maketitle

\pagebreak

\tableofcontents
\listoffigures	

\pagebreak


\section{Einleitung}


Wir machen zeug.

\section{Technik}

F"ur RTL-SDR Projekte werden DVB-T  (Digital Video Broadcasting Terrestrial) Antennen verwendet, welche dann "uber einen USB Dongle mit dem PC verbunden werden \cite{rtlsdr.com}. Der Dongle basiert auf dem RTL2832U Chip, welcher von Realtek entwickelt wird, dieser ist ein leistungsstarker DVB-T Empfangsgleichrichter \cite{realtek}. Mit einem individuell daf"ur entwickelten Treiber k"onnen die Rohdaten im Chip ausgelesen werden. Dabei hat er eine stabile Samplerate von 2,56 MS/s und eine maximale Samplerate von 3,2 MS/s. Bei letzterer k"onnen allerdings einige Samples verloren gehen. Er arbeitet mit einer Aufl"osung von 8 Bits. Dabei weist er eine Eingangsimpedanz von 75 Ohm, bzw. 50 Ohm bei neueren Versionen mit SMA Anschluss. Der RTL-SDR kann Frequenzen zwischen 22 und 2200 MHz verarbeiten \cite{rtlsdr.com}. Das ist besonders wichtig, da der NOAA-18 \footnote{National Oceanic and Atmospheric Administration} Satellit, dessen Signale wir empfangen wollen, mit 137,9125 MHz funkt. Der Satellit befindet sich in einer H"ohe von 854 km und hat damit eine Umlaufzeit von 102 Minuten \cite{osp.noaa}.

\pagebreak

\section{Versuchsbeschreibung}

\section{Versuchsdurchf"uhrung}
\subsection{Ergebnisse}
\section{Fazit}

\pagebreak

\bibliographystyle{acm}
\bibliography{literatur}


\end{document}

